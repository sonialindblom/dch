
\documentclass[12 pt]{paper}
\usepackage[utf8]{inputenc}
\usepackage[american]{babel}
\usepackage{paralist}%in order to enumerate inside a paragraph
%\usepackage[hyphens]{url}
\usepackage[colorlinks=true,linkcolor=blue,anchorcolor=black,citecolor=blue,filecolor=blue,menucolor=black,runcolor=red,urlcolor=blue]{hyperref} %to get hyperlinks
\usepackage{caption}
\usepackage{subcaption}
\usepackage{csquotes}
\newbox{\bigpicturebox}%to create collage in pict 1

\usepackage[style=apa,backend=biber]{biblatex}%to cite using apa7

\usepackage{geometry}%to change page specs
\usepackage{enumitem,varwidth}%image stuff
\usepackage[svgnames]{xcolor}%image stuff, so pictures are colored
%\usepackage{tikz}%image stuff
%\usetikzlibrary{shapes.geometric} %image stuff
%\usepackage{lmodern}
\usepackage{graphicx}%to follow the template from school
\usepackage{booktabs} %to follow the template from school
 \geometry{
	a4paper,
	left=25mm,
	top=25mm,
	right=25mm,
	bottom=25mm,
}
\parindent24pt
%to follow the template from school


\graphicspath{ {./bilder/} } %where the pictures are
\addbibresource{MasterLibrary1.bib}
\title{%
Ditigization of Aerial photograph postcards of Axvall  }

\author{Sonia Lindblom \and Sicheng Yan}

\date{June 2021}

\begin{document}
\begin{titlepage}
	\begin{center}
		\vspace*{1cm}
		
		{\Large \textbf{Collection Development: Aerial postcards}}
		
		\vspace{0.5cm}
	
		
		\vspace{1.5cm}
		
		\textbf{Sonia Lindblom \quad \quad Sicheng Yan}
		
		\vfill
		
		
		
		\vspace{0.8cm}
		
		\includegraphics[width=0.4\textwidth]{1200px-Högskolan_i_Borås_Logo.png}\\
		\vspace{0.4cm}
		Master's programme: Digital Library and Information Services\\
		Swedish School of Library an Information Science \\
		University of Borås\\
		07 June 2021
	
		
	\end{center}
\end{titlepage}

\section{Introduction}
This report of a digitization project of older postcards featuring aerial photos of Axvall. -- insert dizitization in a bigger contexts ---. The aim of this project is to make the images available online to those interested in historical aerial photographs, for instance 
\begin{inparaenum}[i)]
	\item for geomorphology purposes \autocite[cf.][]{gomez2015}
	\item post card historical purposes
	\item history of the village
	\item military history
\end{inparaenum}
\section{Theoretical background}
\subsection{On digitization of cultural heritage}

\autocite{manning2008}
\section{Methodological work}%this has to have about 1300 words)
\subsection{Selection of the material}

The postcards were in shoeboxes without any systematization. Some of them were bound with rubberbands in bunts of 20. The conditions varied from untouched cards to others played with by kids who painted on with crayon or something similar. 
\subsection{Metadata}
\subsection{Image capture}
\subsection{.... platform}% Or something like webdesign

\section {Critical Analysis}


\section{Conclusions}


\newpage

%\bibliographystyle{apacite}
%\bibliography{../../../MasterLibrary.bib}
\printbibliography


\end{document}


